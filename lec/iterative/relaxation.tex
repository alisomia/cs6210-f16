\section{General relaxations}

So far, we have mostly discussed stationary methods in we think of
sweeping through all the variables in some fixed order and
updating a variable or block of variables at a time.  There is nothing
that say that the order must be {\em fixed}, though, if we are willing
to forgo the analytical framework of splittings.  There are essentially
two reasons that we might think to do this:
\begin{enumerate}
\item
  We decide which variable(s) to update next based on some adaptive
  policy, such as which equations have the largest residual.  This
  leads to the {\em Gauss-Southwell} method.  Various methods for fast
  (sublinear time) personalized PageRank use this strategy.
\item
  We update variable(s) on multiple processors, communicating the
  changes opportunistically.  In this case, there may be no real rhyme
  or reason to the order in which we see updates.  These methods are
  called {\em chaotic relaxation} or {\em asynchronous relaxation}
  approaches, and they have seen a great deal of renewed interest over
  the past several years for both classical scientific computing problems
  (e.g.~PDE solvers) and for machine learning applications.
\end{enumerate}
