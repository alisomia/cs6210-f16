\section{Chebyshev: Uses and Limitations}

We previously sketched out an approach for analyzing the convergence of
methods based on Krylov subspaces:
\begin{enumerate}
\item
  Characterize the Krylov subspace of interest in terms of polynomials,
  i.e. $\mathcal{K}_k(A,b) = \{ p(A)b : p \in \mathcal{P}_{k-1} \}$.
\item
  For $\hat{x} = p(A) b$, write an associated error (or residual)
  in terms of a related polynomial in $A$.
\item
  Phrase the problem of minimizing the error, residual, etc.~in terms
  of minimizing a polynomial $q(z)$ on the spectrum of $A$
  (call this $\Lambda(A)$).  The polynomial $q$ must generally satisfy
  some side constraints that prevent the zero polynomial from being
  a valid solution.
\item
  Let $\Lambda(A) \subset \Omega$, and write
  \[
    \max_{\lambda \in \Lambda(A)} |q(\lambda)| \leq
    \max_{z \in Omega} |q(z)|.
  \]
  The set $\Omega$ should be simpler to work with than the set of
  eigenvalues.  The simplest case is when $A$ is symmetric positive
  definite and $\Omega = [\lambda_1, \lambda_n]$.
\item
  The optimization problem can usually be phrased in terms of special
  polynomial families.  The simplest case, when $\Omega$ is just an
  interval, usually leads to an analysis via Chebyshev polynomials.
\end{enumerate}
The analysis sketched above is the basis for the convergence analysis
of the Chebyshev semi-iteration, the conjugate gradient method, and
(with various twists) several other Krylov subspace methods.

The advantage of this type of analysis is that it leads to convergence
bounds in terms of some relatively simple property of the matrix, such
as the condition number.  The disadvantage is that the approximation of
the spectral set $\Lambda(A)$ by a bounding region $\Omega$ can lead to
rather pessimistic bounds.  In practice, the extent to which we are able
to find good solutions in a Krylov subspace often depends on the
``clumpiness'' of the eigenvalues.  Unfortunately, this ``clumpiness''
is rather difficult to reason about a priori!  Thus, the right way to evaluate
the convergence of Krylov methods in practice is usually to try them out,
plot the convergence curves, and see what happens.
