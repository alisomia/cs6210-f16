\section{Spectral transformation}

We have dealt with the notion of spectral transformation before,
when we discussed the power iteration.  The idea of spectral
transformation is to work not with $A$, but with some rational
$f(A)$ where $f$ maps the eigenvalues of interest to the outside
of the spectrum.  Usually $f$ is a rational function; common
examples include
\begin{itemize}
\item {\em Shift-invert}: $f(z) = (z-\sigma)^{-1}$.  Favors
  eigenvalues close to the shift $\sigma$.
\item {\em Cayley}: $f(z) = (\sigma-z) (\sigma +z)^{-1}$.  This maps
  the left half plane to the interior of the unit circle and the
  right half plane to the exterior; it is commonly used in stability
  analysis.
\item {\em Polynomial}: Just what it sounds like.
\end{itemize}
In general, the shifted linear solves needed to carry out rational spectral
transformations (e.g. shift-invert and Cayley) must be computed to
rather high accuracy.  Hence, we favor sparse direct methods.  An alternate
approach, similar to what we say when we briefly considered flexible GMRES,
is to break out of the confines of using a Krylov subspace; the most
popular variant here is the {\em Jacobi-Davidson} method.
