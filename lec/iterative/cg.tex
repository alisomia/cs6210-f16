\section{Conjugate gradients}

We now turn to the method of conjugate gradients (CG), perhaps the
best known of the Krylov subspace solvers.  The CG iteration can be
characterized as the iteration for a symmetric positive definite
$A$ that minimizes the energy
\[
  \phi(x) = \frac{1}{2} x^T A x - x^T b
\]
over a Krylov subspace; as we have already seen,
\[
  \phi(x) + \frac{1}{2} b^T A^{-1} b
  = \frac{1}{2} \|x-A^{-1} b\|_A^2
  = \frac{1}{2} \|Ax-b\|_{A^{-1}}^2,
\]
so this minimization corresponds to minimizing the error in the $A$-norm
or the residual in the $A^{-1}$ norm.  We also have seen the shape of
the standard error analysis, which involves looking at a Chebyshev
polynomial on an interval containing the spectrum.  The iteration turns
out to be forward unstable, so the behavior in floating point arithmetic
is not the same as the behavior in theory; but this does not prevent
the iteration from being highly effective, partly because we can
write the iteration in a form that involves an explicit residual,
and looking at a freshly-computed residual gives the method a
self-correcting property.

Our goal for today is to look at the mechanics of the method.

\subsection{CG via Lanczos}

Last time, we discussed the Lanczos iteration, which produces the
Lanczos decomposition
\[
  A Q_k = Q_{k+1} \bar{T}_k
\]
via the iteration
\[
  \beta_k q_{k+1} = A q_k - \alpha_k q_k - \beta_{k-1} q_{k-1}
\]
where $\alpha_k = q_k^T A q_k$.  One of the simplest derivations
for the {\em conjugate gradient} (CG) method is in terms of the
Lanczos decomposition.

In terms of the energy
\[
  \phi(x) = \frac{1}{2} x^T A x - x^T b,
\]
the problem of finding the ``best'' (minimum energy) approximate
solution in the space becomes
\[
  \mbox{minimize } \phi(Q_k y_k) = \frac{1}{2} y_k^T T_k y_k - y_k^T e_1 \|b\|,
\]
which is solved by
\[
  T_k y_k = e_1 \|b\|.
\]

Now let us suppress the indices for a moment and
write $T = LU$ (which can be computed stably without pivoting,
as $T$ is SPD).  Then we can write the approximate solution
$\hat{x}$ as
\[
  \hat{x} = Q U^{-1} L^{-1} e_1 \|b\|,
\]
which we will group as
\[
  \hat{x} = V \hat{y}, \quad VU = Q, \quad Ly = e_1 \|b\|.
\]
Solving the system for $y$ by forward substitution yields
\begin{align*}
  y_1 &= \|b\| \\
  y_k &= -l_{k,k-1} y_{k-1}.
\end{align*}
Similarly, we can compute the columns of $V$ by forward
substitution:
\begin{align*}
  v_1 &= q_1 / u_{11} \\
  v_k &= \frac{1}{u_{kk}} \left( q_k - v_{k-1} u_{k-1,k} \right).
\end{align*}
The advantage of this formulation is that if we extend the Krylov
subspace, we simply extend the tridiagonal (and associated
factorization), add another component to $y$, and bring in a new
vector $v$ --- all without disturbing the computations done before.
Hence, we have a sequence of coupled recurrences for the columns
of $Q$ and of $V$ that allow us to incrementally update the solution
at the cost of a matrix-vector multiply and a constant amount
of vector arithmetic per step.

This is a useful approach, but it does not shed much insight into how the
method could be extended to optimize more general objectives than
quadratics.  For that, we need the approach that gives the CG method its
name.

\subsection{Another approach to CG}

An alternate approach to the conjugate gradient method does not directly
invoke Lanczos, but instead relies on properties that must be satisfied
at each step by the residual $r_m = b-Ax_m$ and the update
$d_m = x_{m+1}-x_m$.  We assume throughout that $x_m$ is drawn from
$\calK_m(A,b)$, which implies that $r_m \in \calK_{m+1}(A,b)$ and
$d_m \in \calK_{m+1}(A,b)$.

First, note that $r_m \perp \calK_{m}(A,b)$ and $d_m \perp_A
\calK_{m}(A,b)$.\footnote{
  $u \perp_A v$ means $u$ and $v$ are orthogonal
  in the $A$-induced inner product, i.e. $u^T A v = 0$.
}  The former statement comes from the Galerkin
criterion in the previous section.  The latter statement comes from
recognizing that $r_{m+1} = Ad_m + r_m \perp \calK_{m}(A,b)$; with
Galerkin condition $r_m \perp \calK_{m}(A,b)$, this means $Ad_m \perp
\calK_{m}(A,b)$.  Together, these statements give us $r_m$ and $d_m$
to within a scalar factor, since there is only one direction in
$\calK_{m+1}(A,b)$ that is orthogonal to all of $\calK_{m}(A,b)$, and
similarly there is only one direction that is $A$-orthogonal.
This suggests the following idea to generate the sequence of
approximate solutions $x_k$:
\begin{enumerate}
\item
  Find a direction $p_{k-1} \in \calK_{k}(A,b)$ that is $A$-orthogonal
  to $\calK_{k-1}(A,b)$.
\item
  Compute $x_{k} = x_{k-1} + \alpha_k p_{k-1}$ so that
  \[
    r_k = r_{k-1} - \alpha_k A p_{k-1} \perp r_{k-1},
  \]
  i.e. set $\alpha_k = (r_{k-1}^T r_{k-1}) / (p_{k-1}^T A p_{k-1})$.
  Orthogonality to the rest of $\calK_{k}(A,b)$ follows
  automatically from the construction.
\item
  Take $r_k \in \calK_{k+1}(A,b)$ and $A$-orthogonalize against
  everything in $\calK_{k}(A,b)$ to generate the new direction $p_k$.
  As with the Lanczos procedure, the real magic in this idea is that
  we have to do very little work to generate $p_k$ from $r_k$.
  Note that for any $j < k-1$, we have $p_j^T A r_k = (Ap_j)^T r_k = 0$,
  because $Ap_j \in \calK_{j+2}(A,b) \subset \calK_k(A,b)$ is automatically
  orthogonal to $r_k$.  Therefore, we really only need to choose
  \[
    p_k = r_k + \beta p_{k-1},
  \]
  such that $p_{k-1}^T A p_k$, i.e.
  $\beta_k = -(p_{k-1}^T A r_k)/(p_{k-1}^T A p_{k-1})$.
  Note, though, that $Ap_{k-1} = -(r_{k}-r_{k-1})/\alpha_k$;
  with a little algebra, we find
  \[
    \beta_k = -\frac{r_k^T Ap_k}{p_{k-1}^T A p_{k-1}}
            = \frac{(r_k^T r_k)/\alpha_k}{r_{k-1}^T r_{k-1}/\alpha_k}
            = \frac{r_k^T r_k}{r_{k-1}^T r_{k-1}}.
  \]
\end{enumerate}

Putting everything together, we have the following coupled recurrences
for the solutions $x_k$, residuals $r_k$, and search directions $p_k$:
\begin{align*}
  \alpha_k &= (r_{k-1}^T r_{k-1})/(p_{k-1}^T A p_{k-1}) \\
  x_{k} &= x_{k-1} + \alpha_k p_{k-1} \\
  r_{k} &= r_{k-1} - \alpha_k A p_{k-1} \\
  \beta_{k} &= (r_k^T r_k)/(r_{k-1}^T r_{k-1}) \\
  p_k &= r_{k} + \beta_k p_{k-1}.
\end{align*}
The sequences $r_k$ and $p_k$ respectively form orthogonal and $A$-orthogonal
bases for the nested Krylov subspaces generated by $A$ and $b$.

\subsection{Preconditioning}

What happens if we want to compute not on the space $\mathcal{K}_k(A,b)$,
but the preconditioned space $\mathcal{K}_k(M^{-1} A, M^{-1} b)$
where $M$ is some symmetric positive definite matrix?  Unfortunately,
we cannot apply CG directly to a system involving $M^{-1} A$, since
even if $M$ and $A$ are SPD, the product will generally not be.
On the other hand, we can certainly work with the related system
\[
  (M^{-1/2} A M^{-1/2}) (M^{1/2} x) = M^{-1/2} b.
\]
This is a symmetric positive definite system, and the eigenvalues
of $M^{-1/2} A M^{-1/2}$ are the same as the generalized eigenvalues
of the pencil $(A,M)$.  Moreover, we can work with this system
{\em implicitly} without ever having to form the awkward square
root.

Define $\bar{p}_k = M^{-1/2} p_k$ and $\bar{r}_k = M^{1/2} r_k$;
then CG iteration on the related system can be rephrased as
\begin{align*}
  \alpha_k &= (\bar{r}_{k-1}^T M^{-1} \bar{r}_{k-1})/(\bar{p}_{k-1}^T A \bar{p}_{k-1}) \\
  x_{k} &= x_{k-1} + \alpha_k \bar{p}_{k-1} \\
  \bar{r}_{k} &= \bar{r}_{k-1} - \alpha_k A \bar{p}_{k-1} \\
  \beta_{k} &= (\bar{r}_k^T M^{-1} \bar{r}_k)/(\bar{r}_{k-1}^T M^{-1} \bar{r}_{k-1}) \\
  \bar{p}_k &= M^{-1} \bar{r}_{k} + \beta_k \bar{p}_{k-1}.
\end{align*}
Because expressions involving $M^{-1}$ and the residual appear throughout,
we introduce a new variable $z_k = M^{-1} r_k$, leading to
\begin{align*}
  \alpha_k &= (\bar{r}_{k-1}^T z_{k-1})/(\bar{p}_{k-1}^T A \bar{p}_{k-1}) \\
  x_{k} &= x_{k-1} + \alpha_k \bar{p}_{k-1} \\
  \bar{r}_{k} &= \bar{r}_{k-1} - \alpha_k A \bar{p}_{k-1} \\
  M z_{k} &= r_k \\
  \beta_{k} &= (\bar{r}_k^T z_k)/(\bar{r}_{k-1}^T z_{k-1}) \\
  \bar{p}_k &= z_{k} + \beta_k \bar{p}_{k-1}.
\end{align*}

Another way of thinking about the preconditioned CG iteration is
that it is ordinary CG, whether thought of in terms of conjugate
directions or in terms of Lanczos, but with a different inner
product: the $M^{-1}$ inner product on residuals, or the $M$ inner
product in the Lanczos procedure.

\subsection{Nonlinear CG}

One of the advantages of the interpretation of CG in terms of search
directions and residuals is that it generalizes beyond the case of
quadratic optimization or linear system solving to more general
optimization problems.  To derive nonlinear CG, we generalize the
quantities in the ordinary CG iteration in the following way:
\begin{itemize}
\item
  In ordinary CG, we let $\phi$ be a quadratic energy function.
  In nonlinear CG, $\phi$ is a more general (though ideally
  convex) objective function.
\item
  In ordinary CG, we have $r_k = -\nabla \phi(x_k) = b-Ax_k$.
  In nonlinear CG, we take $r_k = -\nabla \phi(x_k)$, though the
  gradient expression will generally be more complicated.
\item
  In ordinary CG, we choose a search direction $p_k = r_k + \beta_k p_{k-1}$
  where $\beta_k = r_k^T r_k / r_{k-1}^T r_{k-1}$.  In nonlinear CG,
  we may use the same formula (the {\em Fletcher-Reeves} formula),
  or we may choose any number of other formulas that are equivalent
  in the quadratic case but not in the more general case.
\item
  In ordinary CG, once we choose a search direction $p_{k-1}$, we compute
  a step $x_k = x_{k-1} + \alpha_{k} p_{k-1}$.  The $\alpha_k$ has
  the property
  \[
    \alpha_k = \operatorname{argmin}_{\alpha} \phi(x_k + \alpha p_{k-1})
  \]
  In nonlinear CG, we instead use a line search to choose the step size.
\end{itemize}
Like ordinary CG, nonlinear CG iterations can be preconditioned.

\subsection{The many approaches to CG}

The description I have given in these notes highlights (I hope)
how orthogonality of the residuals and $A$-orthogonality
of search directions follows naturally from the Galerkin condition,
and how the rest of the CG iteration can be teased out of these
orthogonality relations.  However, this is far from the only way
to ``derive'' the method of conjugate gradients.  The discussion
given by Demmel and by Saad
(in {\em Iterative Methods for Sparse Linear Systems})
highlights the Lanczos connection, and uses this connection to
show the existence of $A$-orthogonal search directions.
Golub and Van Loan show the Lanczos connection, but also show
how conjugate gradients can be derived as a general-purpose minimization
scheme applied to the quadratic function $\phi(x)$.  Trefethen and Bau
give the iteration without derivation first, and then gradually explain
some of its properties.  If you find these discussions confusing,
or simply wish to read something amusing, I recommend Shewchuk's
``\href{http://www.cs.cmu.edu/~quake-papers/painless-conjugate-gradient.pdf}{Introduction to the Conjugate Gradient Method Without the Agonizing Pain}''.
