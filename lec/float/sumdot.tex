\section{Sums and dots}

We already described a couple of floating point examples that
involve evaluation of a fixed formula (e.g.~computation of
the roots of a quadratic).  We now turn to the analysis of
some of the building blocks for linear algebraic computations:
sums and dot products.

\subsection{Sums two ways}

As an example of first-order error analysis, consider the following
code to compute a sum of the entries of a vector $v$:
\begin{lstlisting}
  s = 0;
  for k = 1:n
    s = s + v(k);
  end
\end{lstlisting}
Let $\hat{s}_k$ denote the computed sum at step $k$ of the loop;
then we have
\begin{align*}
  \hat{s}_1 &= v_1 \\
  \hat{s}_k &= (\hat{s}_{k-1} + v_k)(1 + \delta_k), \quad k > 1.
\end{align*}
Running this forward gives
\begin{align*}
  \hat{s}_2 &= (v_1 + v_2)(1+\delta_2) \\
  \hat{s}_3 &= ((v_1 + v_2)(1+\delta_2) + v_3)(1+\delta_2)
\end{align*}
and so on.  Using first-order analysis, we have
\[
  \hat{s}_k \approx (v_1 + v_2)\left(1 + \sum_{j=2}^k \delta_j \right)
              + \sum_{l=3}^k v_l \left( 1 + \sum_{j=l}^k \delta_j \right),
\]
and the difference between $\hat{s}_k$ and the exact partial sum
is then
\[
  \hat{s}_k-s_k \approx \sum_{j=2}^k s_j \delta_j.
\]
Using $\|v\|_1$ as a uniform bound on all the partial sums, we have
\[
  |\hat{s}_n-s_n| \lesssim (n-1) \macheps \|v\|_2.
\]

An alternate analysis, which is a useful prelude to analyses to come
involves writing an error recurrence.
Taking the difference between $\hat{s}_k$ and the true partial sums $s_k$,
we have
\begin{align*}
  e_1 &= 0 \\
  e_{k} &= \hat{s}_k-s_k \\
        &= (\hat{s}_{k-1} + v_k)(1+\delta_k) - (s_{k-1} + v_k) \\
        &= e_{k-1} + (\hat{s}_{k-1} + v_k) \delta_k,
\end{align*}
and $\hat{s}_{k-1} + v_k = s_k + O(\macheps)$, so that
\[
  |e_{k}| \leq |e_{k-1}| + |s_k| \macheps + O(\macheps^2).
\]
Therefore,
\[
  |e_{n}| \lesssim (n-1) \macheps \|v\|_1,
\]
which is the same bound we had before.

\subsection{Backward error analysis for sums}

In the previous subsection, we showed an error analysis for partial sums
leading to the expression:
\[
  \hat{s}_n \approx (v_1 + v_2) \left(1 + \sum_{j=2}^n \delta_j \right)
              + \sum_{l=3}^n v_l \left( 1 + \sum_{j=l}^n \delta_j \right).
\]
We then proceded to aggregate all the rounding error terms in order to
estimate the error overall.  As an alternative to aggregating the roundoff,
we can also treat the rounding errors as perturbations to the input
variables (the entries of $v$); that is, we write the computed sum as
\[
  \hat{s}_n = \sum_{j=1}^n \hat{v}_j
\]
where
\[
  \hat{v}_j = v_j (1+\eta_j), \quad
  \mbox{ where }|\eta_j| \lesssim (n+1-j) \macheps.
\]
This gives us a {\em backward error} formulation of the rounding:
we have re-cast the role of rounding error in terms of a
perturbation to the input vector $v$.  In terms of the 1-norm,
we have the relative error bound
\[
  \|\hat{v}-v\|_1 \lesssim n \macheps \|v\|_1;
\]
or we can replace $n$ with $n-1$ by being a little more careful.
Either way, what we have shown is that the summation algorithm
is {\em backward stable}, i.e.~we can ascribe the roundoff to
a (normwise) small relative error with a bound of $C \macheps$
where the constant $C$ depends on the size $n$ like some low-degree
polynomial.

Once we have a bound on the backward error, we can bound the forward
error via a condition number.  That is, suppose we write the true
and perturbed sums as
\begin{align*}
  s &= \sum_{j=1}^n v_j &
  \hat{s} &= \sum_{j=1}^n \hat{v}_j.
\end{align*}
We want to know the relative error in $\hat{s}$ via a normwise
relative error bound in $\hat{v}$, which we can write as
\[
  \frac{|\hat{s}-s|}{|s|} =
  \frac{|\sum_{j=1}^n (\hat{v}_j-v_j)|}{|s|} \leq
  \frac{\|\hat{v}-v\|_1}{|s|} =
  \frac{\|v\|_1}{|s|} \frac{\|\hat{v}-v\|_1}{\|v\|_1}.
\]
That is, $\|v\|_1/|s|$ is the condition number for the summation
problem, and our backward stability analysis implies
\[
  \frac{|\hat{s}-s|}{|s|} \leq
  \frac{\|v\|_1}{|s|} n \macheps.
\]
This is the general pattern we will see again in the future:
our analysis consists of a backward error computation that depends
purely on the algorithm, together with a condition number that depends
purely on the problem.  Together, these give us forward error bounds.

\subsection{Running error bounds for sums}

In all the analysis of summation we have done so far, we ultimately
simplified our formulas by bounding some quantity in terms of $\|v\|_1$.
This is nice for algebra, but we lose some precision in the process.
An alternative is to compute a {\em running error bound}, i.e.~augment
the original calculation with something that keeps track of the error
estimates.  We have already seen that the error in the computations
looks like
\[
  \hat{s}_n-s_n = \sum_{j=2}^n s_j \delta_j + O(\macheps^2),
\]
and since $s_j$ and $\hat{s}_j$ differ only by $O(\macheps)$ terms,
\[
  |\hat{s}_n-s_n| \lesssim \sum_{j=2}^n |\hat{s}_j| \macheps + O(\macheps^2),
\]
We are not worried about doing a rounding error analysis of our rounding
error analysis --- in general, we care more about order of magnitude for
rounding error anyhow --- so the following routine does an adequate job
of computing an (approximate) upper bound on the error in the summation:
\begin{lstlisting}
  s = 0;
  e = 0;
  for k = 1:n
    s = s + v(k);
    e = e + abs(s) * eps;
  end
\end{lstlisting}

\subsection{Compensated summation}

We conclude our discussion of rounding analysis for summation
with a comment on the {\em compensated summation} algorithm of
Kahan, which is not amenable to straightforward $1+\delta$ analysis.
The algorithm maintains the partial sums not as a single variable
{\tt s}, but as an unevaluated sum of two
variables {\tt s} and {\tt c}:
\begin{lstlisting}
  s = 0;
  c = 0;
  for k = 1:n
    y = v(i) - c;
    t = s + y;
    c = (t - s) - y;  % Key step
    s = t;
  end
\end{lstlisting}
Where the error bound for ordinary summation is
$(n-1) \macheps \|v\|_1 + O(\macheps^2)$,
the error bound for compensated summation is
$2 \macheps \|v\|_1 + O(\macheps^2)$.  Moreover,
compensated summation is exact for adding up to $2^k$
terms that are within about $2^{p-k}$ of each other
in magnitude.

Nor is Kahan's algorithm the end of the story!
Higham's {\em Accuracy and Stability of Numerical Methods}
devotes an entire chapter to summation methods, and there
continue to be papers written on the topic.  For our purposes,
though, we will wrap up here with two observations:
\begin{itemize}
\item Our initial analysis in the $1+\delta$ model illustrates the
  general shape these types of analyses take and how we can re-cast
  the effect of rounding errors as a ``backward error'' that perturbs
  the inputs to an exact problem.
\item The existence of algorithms like Kahan's compensated summation
  method should indicate that the backward-error-and-conditioning
  approach to rounding analysis is hardly the end of the story.
  One could argue it is hardly the beginning!  But it is the approach
  we will be using for most of the class.
\end{itemize}

\subsection{Dot products}

We conclude with one more example error analysis, this time involving
a real dot product computed by a loop of the form
\begin{lstlisting}
  dot = 0;
  for k = 1:n
    dot = dot + x(k)*y(k);
  end
\end{lstlisting}
Unlike the simple summation we analyzed above, the dot product involves
two different sources of rounding errors: one from the summation,
and one from the product.  As in the case of simple summations, it is
convenient to re-cast this error in terms of perturbations to the input.
We could do this all in one go, but since we have already spent so much
time on summation, let us instead do it in two steps.  Let $v_k = x_k y_k$;
in floating point, we get $\hat{v}_k = v_k (1+\eta_k)$
where $|\eta_k| < \macheps$.  Further,
we have already done a backward error analysis of summation to show
that the additional error in summation can be cast onto the summands,
i.e.~the floating point result is $\sum_k \tilde{v}_k$ where
\begin{align*}
  \tilde{v}_k
  &= \hat{v}_k (1 + \sum_{j=\min(2,n)}^n \delta_j) (1 + \eta_k) + O(\macheps^2) \\
  &= v_k (1+\gamma_k) + O(\macheps^2)
\end{align*}
where
\[
  |\gamma_k| = |\eta_k + \sum_{j=\min(2,n)}^n \delta_j| \leq n \macheps.
\]
Rewriting $v_k(1+\gamma_k)$ as $\hat{x}_k y_k$ where
$\hat{x}_k = x_k (1+\gamma_k)$, we have that the computed inner
product $y^T x$ is equivalent to the exact inner product of $y^T \hat{x}$
where $\hat{x}$ is an elementwise relatively accurate
(to within $n \macheps + O(\macheps^2)$) approximation to $x$.

A similar backward error analysis shows that computed matrix-matrix
products $AB$ in general can be interpreted as $\hat{A} B$ where
\[
  |\hat{A}-A| < p \macheps |A| + O(\macheps^2)
\]
and $p$ is the inner dimension of the product.  Exactly what $\hat{A}$
is depends not only on the data, but also the loop order used in the
multiply --- since, as we recall, the order of accumulation may vary
from machine to machine depending on what blocking is best suited to
the cache!  But the bound on the backward error holds for all the common
re-ordering\footnote{%
For those of you who know about Strassen's algorithm --- it's not
backward stable, alas.
}
And this backward error characterization, together with the
type of sensitivity analysis for matrix multiplication that we have
already discussed, gives us a uniform framework for obtaining
forward error bounds for matrix-matrix muliplication; and the same
type of analysis will continue to dominate our discussion of rounding
errors as we move on to more complicated matrix computations.
