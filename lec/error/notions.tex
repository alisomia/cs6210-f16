\section{Notions of error}

The art of numerics is finding an approximation with a fast algorithm,
a form that is easy to analyze, and an error bound.  Given a task, we
want to engineer an approximation that is good enough, and that
composes well with other approximations.  To make these goals precise,
we need to define types of errors and error propagation, and some
associated notation -- which is the point of this lecture.

\subsection{Absolute and relative error}

Suppose $\hat{x}$ is an approximation to $x$.
The {\em absolute error} is
\[
  e_{\mathrm{abs}} = |\hat{x}-x|.
\]
Absolute error has the same dimensions as $x$,
and can be misleading without some context.  An error
of one meter per second is dramatic if $x$ is my walking
pace; if $x$ is the speed of light, it is a very small error.

The {\em relative error} is a measure with a more natural
sense of scale:
\[
  e_{\mathrm{rel}} = \frac{|\hat{x}-x|}{|x|}.
\]
Relative error is familiar in everyday life: when someone
talks about an error of a few percent, or says that a given
measurement is good to three significant figures, she is describing
a relative error.

We sometimes estimate the relative error in approximating
$x$ by $\hat{x}$ using the relative error in approximating
$\hat{x}$ by $x$:
\[
  \hat{e}_{\mathrm{rel}} = \frac{|\hat{x}-x|}{|\hat{x}|}.
\]
As long as $\hat{e}_{\mathrm{rel}} < 1$, a little algebra gives that
\[
  \frac{\hat{e}_{\mathrm{rel}}}{1+\hat{e}_{\mathrm{rel}}} \leq
  e_{\mathrm{rel}} \leq
  \frac{\hat{e}_{\mathrm{rel}}}{1-\hat{e}_{\mathrm{rel}}}.
\]
If we know $\hat{e}_{\mathrm{rel}}$ is much less than one, then it
is a good estimate for $e_{\mathrm{rel}}$.  If
$\hat{e}_{\mathrm{rel}}$ is not much less than one,
we know that $\hat{x}$ is a poor approximation to $x$.
Either way, $\hat{e}_{\mathrm{rel}}$ is often just as useful
as $e_{\mathrm{rel}}$, and may be easier to estimate.

Relative error makes no sense for $x = 0$, and may be too pessimistic
when the property of $x$ we care about is ``small enough.''  A natural
intermediate between absolute and relative errors is the mixed error
\[
  e_{\mathrm{mixed}} = \frac{|\hat{x}-x|}{|x| + \tau}
\]
where $\tau$ is some natural scale factor associated with $x$.

\subsection{Errors beyond scalars}

Absolute and relative error make sense for vectors as well as scalars.
If $\| \cdot \|$ is a vector
norm and $\hat{x}$ and $x$ are vectors, then the (normwise) absolute
and relative errors are
\begin{align*}
  e_{\mathrm{abs}} &= \|\hat{x}-x\|, &
  e_{\mathrm{rel}} &= \frac{\|\hat{x}-x\|}{\|x\|}.
\end{align*}
We might also consider the componentwise absolute or relative errors
\begin{align*}
  e_{\mathrm{abs},i} &= |\hat{x}_i-x_i| &
  e_{\mathrm{rel},i} &= \frac{|\hat{x}_i-x_i|}{|x_i|}.
\end{align*}
The two concepts are related: the maximum componentwise relative error
can be computed as a normwise error in a norm defined in terms of the
solution vector:
\begin{align*}
  \max_i e_{\mathrm{rel},i} &= \vertiii{\hat{x}-x}
\end{align*}
where $\vertiii{z} = \|\ddiag(x)^{-1} z\|$.
More generally, absolute error makes sense whenever we can measure
distances between the truth and the approximation; and relative error
makes sense whenever we can additionally measure the size of the
truth.  However, there are often many possible notions of distance
and size; and different ways to measure give different notions of
absolute and relative error.  In practice, this deserves some care.

\subsection{Dimensions and scaling}

The first step in analyzing many application problems is
{\em nondimensionalization}: combining constants in the
problem to obtain a small number of dimensionless constants.
Examples include the aspect ratio of a rectangle,
the Reynolds number in fluid mechanics\footnote{%
Or any of a dozen other named numbers in fluid mechanics.  Fluid
mechanics is a field that appreciates the power of dimensional
analysis}, and so forth.  There are three big reasons to
nondimensionalize:
\begin{itemize}
\item
  Typically, the physics of a problem only really depends on
  dimensionless constants, of which there may be fewer than
  the number of dimensional constants.  This is important
  for parameter studies, for example.
\item
  For multi-dimensional problems in which the unknowns have different
  units, it is hard to judge an approximation error as ``small'' or
  ``large,'' even with a (normwise) relative error estimate.  But one
  can usually tell what is large or small in a non-dimensionalized
  problem.
\item
  Many physical problems have dimensionless parameters much less than
  one or much greater than one, and we can approximate the physics in
  these limits.  Often when dimensionless constants are huge or tiny
  and asymptotic approximations work well, naive numerical methods
  work work poorly.  Hence, nondimensionalization helps us choose how
  to analyze our problems --- and a purely numerical approach may be
  silly.
\end{itemize}
