\section{The 2D model problem}

The problem with the 1D Poisson equation is that it doesn't make a
terribly convincing challenge -- since it is a symmetric positive
definite tridiagonal, we can solve it in linear time with Gaussian
elimination!  So let us turn to a slightly more complicated example:
the Poisson equation in 2D.  Before discussing the 2D Poisson
equation, though, let us digress to introduce two useful notations:
the $\operatorname{vec}$ operator and the Kronecker product.

The $\operatorname{vec}$ operator simply lists the entries of a matrix (or an
array with more than two indices) in column-major order; for example,
\[
  \operatorname{vec} \begin{bmatrix} a & b \\ c & d \end{bmatrix} =
  \begin{bmatrix} a \\ c \\ b \\ d \end{bmatrix}.
\]
The Kronecker product $A \otimes B$ of two matrices is a block matrix
where each block is a scalar multiple of $B$:
\[
  A \otimes B =
  \begin{bmatrix}
    a_{11} B & a_{12} B & \ldots \\
    a_{21} B & a_{22} B & \ldots \\
    \vdots & \vdots & \ddots
  \end{bmatrix}
\]
The Kronecker product and the vec operation interact with each other
as follows:
\[
  (B \otimes A) \operatorname{vec}(C) = \operatorname{vec}(ACB^T).
\]
The Kronecker product also satisfies the identities
\begin{align*}
  (A \otimes B)^T &= A^T \otimes B^T \\
  (A \otimes B) (C \times D) &= (AB) \otimes (CD)
\end{align*}
which implies, for example, that the Schur form of a Kronecker product
is a Kronecker product of Schur forms:
\[
  (U_A \otimes U_B)^* (A \otimes B) (U_A \otimes U_B)= T_A \otimes T_B.
\]

As one illustrative application of Kronecker products, consider the
Sylvester operator $X \mapsto AX-XB$.  Using Kronecker products, we
can write this as
\[
  \operatorname{vec}(AX-XB) =
  (A \otimes I - I \otimes B) \operatorname{vec}(X).
\]
Note that if $A = U_A T_A U_A^*$ and $B = U_B T_B U_B^*$ are Schur
forms, then
\[
  A \otimes I - I \otimes B =
  (U_A \otimes U_B) (T_A \otimes I - I \otimes T_B) (U_A \otimes U_B)^*,
\]
and $T_A \otimes I - T_B \otimes I$ is an upper triangular matrix.
This transformation, followed by a triangular solve, is essentially
what you did in problem 3 of your last homework.

Now let us return to the model 2D Poisson discretization.  This is an
approximation to the equation
\[
  -\nabla^2 u =
  -\left(
  \frac{\partial^2 u}{\partial x^2}+
  \frac{\partial^2 u}{\partial y^2} \right)
  = f
\]
for $(x,y) \in (0,1)^2$, with Dirichlet boundary conditions $u(x,y) =
0$ for $|x| = 1$ or $|y| = 1$.  If we discretize on a regular mesh with
interior points indexed by $1 \leq i \leq n$ and $1 \leq j \leq n$,
we can write the solution as a matrix $U$.  When we discretize, we
have a partial derivative in $x$ corresponding to acting across
columns of $U$, and a partial derivative in $y$ corresponding to
acting across rows of $U$.  We can write this operation as
\[
  TU + UT = h^2 F,
\]
or as an ordinary matrix equation of dimension $N = n^2$
\[
  (T \otimes I + I \otimes T) \operatorname{vec}(U) = h^2 \operatorname{vec}(F).
\]

What properties do we have for $T_{n \times n} = T \otimes I + I
\otimes T$?
\begin{enumerate}
\item
  $T_{n \times n}$ is symmetric and positive definite.
\item
  $T_{n \times n}$ is (non-strictly) diagonally dominant.
\item
  If $(z_j, \lambda_j)$ are the eigenpairs for $T$, those for
  $T_{n \times n}$ are $(z_j \otimes z_l, \lambda_j+\lambda_l)$.
\item
  The condition number of $T_{n \times n}$ scales like $O(h^{-2})$.
\end{enumerate}
