\section{The 1D model problem}

It is difficult to say many useful
things about the convergence of iterative methods without looking at a
concrete problem.  Therefore, we will set the stage with a very
specific model problem: a discretization of the Poisson equation.
We start with the one-dimensional case.

The continuous version of our model problem is a one-dimensional
Poisson equation with homogeneous Dirichlet boundary conditions:
\begin{align*}
  -\frac{d^2 u}{dx^2} &= f \mbox{ for } x \in (0,1) \\
  u(0) & = 0 \\
  u(1) & = 0
\end{align*}
Let $x_j = j/(n+1)$ for $j = 0, 1, \ldots, n+1$ be a set of mesh points.
We can approximate the second derivative of $u$ at a point by
a finite difference method:
\[
  -\frac{d^2 u}{dx^2}(x_j) \approx
    \frac{-u(x_{j-1})+2u(x_j)-u(x_{j+1})}{h^2}
\]
where $h = 1/(n+1)$ is the mesh spacing.  If we replace the second
derivative in the Poisson equation with this finite-difference approximation,
we have a scheme for computing $u_j \approx u(x_j)$:
\begin{align*}
  -u_{j-1} + 2u_j - u_{j-1} &= h^2 f_j \mbox{ for } 1 \leq j \leq n \\
  u_0 & = 0 \\
  u_{n+1} & = 0
\end{align*}
We can write this approximation as a matrix equation $Tu = h^2 f$, where
\[
  T = \begin{bmatrix}
      2 & -1 \\
      -1 & 2 & -1 \\
         & -1 & 2 & -1 \\
         &    & \ddots & \ddots & \ddots \\
         &    &        & -1 & 2      & -1 \\
         &    &        &    & -1      & 2
      \end{bmatrix}
\]

Part of what makes this simple Poisson discretization so appealing as
a model problem is that we can compute the eigenvalues and eigenvectors
directly.  This is because solving the $(T-\lambda) \psi = 0$ is
equivalent to considering the constant coefficient difference equation
\[
  \psi_{k+1} - (2-\lambda) \psi_k + \psi_{k-1} = 0
\]
subject to the boundary conditions $\psi_0 = \psi_{n+1} = 0$.  Solutions to
this difference equation must have the form
\[
  \psi_{k} = \alpha \xi^k + \beta \bar{\xi}^k,
\]
where $\xi$ and $\bar{\xi}$ are the roots of the characteristic
polynomial $p(z) = z^2 - (2-\lambda) z + 1$.  For $0 \leq \lambda \leq 4$,
these roots form a complex conjugate pair, each with unit magnitude;
that is, we can write $\xi = \exp(i\theta)$ for some $\theta$, and so
\[
  \xi^k = \exp(ik\theta) = \cos(k\theta) + i \sin(k\theta).
\]
Thus, any solution to the difference equation must have the form
\[
  \psi_{k} = \gamma \cos(k\theta) + \mu \sin(k\theta).
\]
Plugging in the boundary conditions, we find that $\gamma = 0$,
and $\theta = l\pi/(n+1)$ for some $l$.  Thus, the normalized eigenvectors
of $T$ are $z_j$ with entries
\[
  z_j(k) = \sqrt{\frac{2}{n+1}} \sin\left( \frac{jk\pi}{n+1} \right),
         = \sqrt{\frac{2}{n+1}} \sin( (j\pi) x_k )
\]
and the corresponding eigenvalues are
\[
  \lambda_j = 2 \left( 1 - \cos \frac{\pi j}{n+1} \right).
\]
For $j \ll n$, Taylor expansion gives that
\[
  \lambda_j = h^2 (\pi j)^2 + O\left( h^4 (\pi j)^4 \right).
\]

By way of comparison, the continuous Dirichlet eigenvalue
problem
\[
  -\frac{d^2 w}{dx^2} = \mu w, \quad w(0) = w(1) = 0
\]
has eigenfunctions of the form
\[
  w_j = \sin(j\pi x), \quad \mu_j = (j \pi)^2.
\]
Thus, the eigenvectors of $h^{-2} T$ are {\em exactly} the sampled
eigenfunctions of $-d^2/dx^2$ on $[0,1]$ with Dirichlet boundary
conditions, while the extremal eigenvalues of $h^{-2} T$ satisfy
\[
  h^{-2} \lambda_j = \mu_j + O(\mu_j^2 h^2).
\]
