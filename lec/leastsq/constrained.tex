\section{Constrained case}

Consider the weighted least squares problem
\[
  \mbox{minimize } \sum_{i=1}^m w_i r_i^2
\]
where $w_1$ is much larger than the others.  If we
let $w_1 \rightarrow \infty$ while the others are fixed, what happens?
We essentially say that we care about enforcing the first equation
above all others, and in the limit we are solving the {\em constrained}
least squares problem
\[
  \mbox{minimize } \sum_{i=2}^m w_i r_i^2 \mbox{ s.t. } r_1 = 0.
\]
Unfortunately, if we actually try to compute this way, we are dancing on
dangerous ground; as $w_1$ goes to infinity, so does the condition
number of the least squares problem.  But this is only an issue with the
weighted formulation; we can formulate the constrained problem in other
ways that are perfectly well-behaved.

In the remainder of this section, we address two ways of handling
the linearly constrained least squares problem
\[
  \mbox{minimize } \|Ax-b\|^2 \mbox{ s.t. } C^T x = d,
\]
by either eliminating variables (the {\em null-space method}) or adding
variables (the method of {\em Lagrange multipliers}).

\subsection{Null space method}

In the null space method, we write an explicit expression for the solutions
to $C^T x = d$ in the form $x^p + W z$ where $x^p$ is a particular solution
to $C^T x^p = d$ and $W$ is a basis for the null space of $C^T$.  Perhaps the
simplest particular solution is $x = C^\dagger d$; we can compute both
this particular solution and an orthogonormal null space basis quickly
using a full QR decomposition of $C$:
\[
  C =
    \begin{bmatrix} Q_1 & Q_2 \end{bmatrix}
    \begin{bmatrix} R_1 \\ 0 \end{bmatrix}, \quad
  x = R_1^{-1} Q_1^T d, \quad W = Q_2.
\]
Having written an explicit parameterization for all solutions of the
constraint equations, we can minimize the least squares objective with
respect to the reduced set of variables
\[
  \mbox{minimize } \|A(x^p + Wz) - b\|^2 = \|(AW)z - (b-Ax^p)\|^2.
\]
This new least squares problem involves a smaller set of variables
(which is good); but in general, even if $A$ is sparse, $AW$ will not be.
So it is appropriate to have a few more methods in our arsenal.

\subsection{Lagrange multipliers}

An alternate method is the method of {\em Lagrange multipliers}.
This is an algebraic technique for adding equations to enforce constraints.

One way to approach the Lagrange multiplier method is to look at the
equations for a constrained minimum.  In order not to have a downhill
direction, we require that the directional derivatives be zero in any
direction consistent with the constraint; that is, we require $Cx = d$
and
\[
  \delta x^T A^T r = 0 \mbox{ when } C^T \delta x = 0.
\]
The constraint says that admissible $\delta x$ are
orthogonal to the columns of $C$; the objective tells us the admissible
$\delta x$ should be orthogonal to the residual.  So we need that $A^T r$
should lie in the column span of $C$; that is,
\[
  A^T r = -C \lambda
\]
for some $\lambda$, and $Cx = d$.  Putting this together,
we have the KKT equations
\[
  \begin{bmatrix}
    A^T A & C \\
    C^T & 0
  \end{bmatrix}
  \begin{bmatrix} x \\ \lambda \end{bmatrix} =
  \begin{bmatrix} b \\ d \end{bmatrix}.
\]

These bordered normal equations are not the end point for constrained
least squares  with Lagrange multipliers, any more than the normal
equations are the end point for unconstrained least squares.  Rather,
we can use this as a starting point for clever manipulations involving
our favorite factorizations (QR and SVD) that reduce the bordered system
to a more computationally convenient form.
