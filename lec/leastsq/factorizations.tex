\section{A family of factorizations}

\subsection{Cholesky}

If $A$ is full rank, then $A^T A$ is symmetric and positive definite
matrix, and we can compute a Cholesky factorization of $A^T A$:
\[
  A^T A = R^T R.
\]
The solution to the least squares problem is then
\[
  x = (A^T A)^{-1} A^T b = R^{-1} R^{-T} A^T b,
\]
or, in MATLAB world
\begin{lstlisting}
  R = chol(A'*A, 'upper');
  x = R\(R'\(A'*b));
\end{lstlisting}

\subsection{Economy QR}

The Cholesky factor $R$ appears in a different setting as well.
Let us write $A = QR$ where $Q = AR^{-1}$; then
\[
  Q^T Q = R^{-T} A^T A R^{-1} = R^{-T} R^T R R^{-1} = I.
\]
That is, $Q$ is a matrix with orthonormal columns.  This
``economy QR factorization'' can be computed in several different
ways, including one that you have seen before in a different guise
(the Gram-Schmidt process).  MATLAB provides a numerically stable
method to compute the QR factorization via
\begin{lstlisting}
  [Q,R] = qr(A,0);
\end{lstlisting}
and we can use the QR factorization directly to solve the least
squares problem without forming $A^T A$ by
\begin{lstlisting}
  [Q,R] = qr(A,0);
  x = R\(Q'*b);
\end{lstlisting}

\subsection{Full QR}

There is an alternate ``full'' QR decomposition where we write
\[
A = QR, \mbox{ where }
Q = \begin{bmatrix} Q_1 & Q_2 \end{bmatrix} \in \bbR^{n \times n},
R = \begin{bmatrix} R_{1} \\ 0 \end{bmatrix} \in \bbR^{m \times n}.
\]
To see how this connects to the least squares problem, recall
that the Euclidean norm is invariant under orthogonal transformations,
so
\[
  \|r\|^2 = \|Q^T r\|^2 = \left\| \begin{bmatrix} Q_1^T b \\ Q_2^T
    b \end{bmatrix} - \begin{bmatrix} R_1 \\ 0 \end{bmatrix} x
  \right\|^2 = \|Q_1^T b-R_1x\|^2 + \|Q_2^T b\|^2.
\]
We can set $\|Q_1^T v-R_1 x\|^2$ to zero by
setting $x = R_1^{-1} Q_1^T b$; the result is
$\|r\|^2 = \|Q_2^T b\|^2$.

\subsection{SVD}

The full QR decomposition is useful because orthogonal transformations
do not change lengths.  Hence, the QR factorization lets us change
to a coordinate system where the problem is simple without changing
the problem in any fundamental way.  The same is true of the SVD,
which we write as
\begin{align*}
A &=
\begin{bmatrix} U_1 & U_2 \end{bmatrix}
\begin{bmatrix} \Sigma \\ 0 \end{bmatrix}
V^T & & \mbox{Full SVD} \\
&= U_1 \Sigma V^T & & \mbox{Economy SVD}.
\end{align*}
As with the QR factorization, we can apply an orthogonal
transformation involving the factor $U$ that makes the
least squares residual norm simple:
\[
\|U^T r\|^2 =
\left\| \begin{bmatrix} U_1^T b \\ U_2^T b \end{bmatrix} -
\begin{bmatrix} \Sigma V^T \\ 0 \end{bmatrix} x
\right\| =
\|U_1^T b - \Sigma V^T x\|^2 + \|U_2^T b\|^2,
\]
and we can minimize by setting $x = V \Sigma^{-1} U_1^T b$.
