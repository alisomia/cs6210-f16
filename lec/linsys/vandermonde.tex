\section{Vandermonde matrices}

The case of {\em Vandermonde matrices} is interesting for several
reasons:
\begin{itemize}
  \item They are highly structured.
  \item They are horribly conditioned.
  \item The ill-conditioned matrix appears as an intermediate
    in a problem that may be just fine.
\end{itemize}

A Vandermonde matrix is a matrix $V \in \bbR^{n \times n}$
whose entries are
\[
  v_{ij} = \xi_i^{j-1}.
\]
The matrix appears in polynomial interpolation.  The linear
system $Vc = f$ is equivalent to the conditions
\[
  p(\xi_i) = \sum_{j=1}^n c_j \xi_i^{j-1} = f_i.
\]
Assuming the $\xi$ are all distinct, this system is nonsingular.
However, the condition number grows {\em exponentially} as a
function of $n$.  Does this mean that the problem of polynomial
interpolation is horribly ill-conditioned?  Of course not!
The exponential ill-conditioning has to do with the expression
of the polynomial as a linear combination of monomials (the
so-called power basis).   But {\em we don't care} what the coefficient
vector $c$ will be; we just want a representation for the
interpolating polynomial $p$ that we can evaluate at points
other than the $\xi_i$.  If we represent $p$ in a different basis,
we often get a problem that is perfectly well-behaved.
