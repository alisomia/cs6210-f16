\section{Cholesky}

The algorithm to compute the Cholesky factor of an SPD matrix is
close to the Gaussian elimination algorithm.  In the first step,
we would write
\[
  \begin{bmatrix}
    a_{11} & a_{21}^T \\
    a_{21} & A_{22}
  \end{bmatrix} =
  \begin{bmatrix}
    l_{11} & 0 \\
    l_{21} & L_{22}
  \end{bmatrix}
  \begin{bmatrix}
    l_{11} & l_{21}^T \\
        0 & L_{22}^T
  \end{bmatrix},
\]
or
\begin{align*}
  a_{11} &= l_{11}^2 \\
  a_{21} &= l_{21} l_{11} \\
  A_{22} &= L_{22} L_{22}^T + l_{21} l_{21}^T.
\end{align*}
The first two equations allow us to compute the first column of $L$;
the last equation tells us that the rest of $L$ is the Cholesky factor
of a Schur complement, $L_{22} L_{22}^T = A_{22} - l_{21} l_{21}^T$.
Continuing in this fashion, we have the algorithm
\begin{lstlisting}
  for j = 1:n

    % Compute the jth column of L
    A(j,j) = sqrt(A(j,j));
    A(j+1:end,j) = A(j+1:end,j)/A(j,j);

    % Update trailing submatrix
    A(j+1:end,j+1:end) = A(j+1:end,j+1:end) - ...
                         A(j+1:end,j)*A(j+1:end,j)';
  end

  % Lower triangle was overwritten by L
  L = tril(A);
\end{lstlisting}
Like the nearly-identical Gaussian elimination algorithm, we can
rewrite the Cholesky algorithm in block form for better cache use.
Unlike Gaussian elimination, we do just fine using Cholesky {\em without}
pivoting\footnote{Pivoting can still be useful for near-singular matrices,
but unpivoted Cholesky is backward stable}.
