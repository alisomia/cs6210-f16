\section{Gaussian elimination by example}

Let's start our discussion of $LU$ factorization by working through
these ideas with a concrete example:
\[
  A =
  \begin{bmatrix}
    1 & 4 & 7 \\
    2 & 5 & 8 \\
    3 & 6 & 10
  \end{bmatrix}.
\]
To eliminate the subdiagonal entries $a_{21}$ and $a_{31}$, we
subtract twice the first row from the second row, and thrice the
first row from the third row:
\[
  A^{(1)} =
  \begin{bmatrix}
    1 & 4 & 7 \\
    2 & 5 & 8 \\
    3 & 6 & 10
  \end{bmatrix} -
  \begin{bmatrix}
    0 \cdot 1 & 0 \cdot 4 & 0 \cdot 7 \\
    2 \cdot 1 & 2 \cdot 4 & 2 \cdot 7 \\
    3 \cdot 1 & 3 \cdot 4 & 3 \cdot 7
  \end{bmatrix}
  =
  \begin{bmatrix}
    1 &  4 &  7 \\
    0 & -3 & -6 \\
    0 & -6 & -11
  \end{bmatrix}.
\]
That is, the step comes from a rank-1 update to the matrix:
\[
  A^{(1)} =
  A -
  \begin{bmatrix} 0 \\ 2 \\ 3 \end{bmatrix}
  \begin{bmatrix} 1 & 4 & 7 \end{bmatrix}.
\]
Another way to think of this step is as a linear transformation
$A^{(1)} = M_1 A$, where the rows of $M_1$ describe the multiples
of rows of the original matrix that go into rows of the updated matrix:
\[
  M_1 = \begin{bmatrix} 1 & 0 & 0 \\ -2 & 1 & 0 \\ -3 & 0 & 1 \end{bmatrix}
      = I - \begin{bmatrix} 0 \\ 2 \\ 3 \end{bmatrix}
            \begin{bmatrix} 1 & 0 & 0 \end{bmatrix}
      = I - \tau_1 e_1^T.
\]
Similarly, in the second step of the algorithm, we subtract twice the second
row from the third row:
\[
  \begin{bmatrix}
    1 &  4 &  7 \\
    0 & -3 & -6 \\
    0 &  0 &  1
  \end{bmatrix} =
  \begin{bmatrix}
    1 & 0 & 0 \\
    0 & 1 & 0 \\
    0 & -2 & 1
  \end{bmatrix}
  \begin{bmatrix}
    1 &  4 &  7 \\
    0 & -3 & -6 \\
    0 & -6 & -11
  \end{bmatrix} =
  \left( I - \begin{bmatrix} 0 \\ 0 \\ 2 \end{bmatrix}
             \begin{bmatrix} 0 & 1 & 0 \end{bmatrix} \right) A^{(1)}.
\]
More compactly: $U = (I-\tau_2 e_2^T) A^{(1)}$.

Putting everything together, we have computed
\[
  U = (I-\tau_2 e_2^T) (I-\tau_1 e_1^T) A.
\]
Therefore,
\[
  A = (I-\tau_1 e_1^T)^{-1} (I-\tau_2 e_2^T)^{-1} U = LU.
\]
Now, note that
\[
  (I-\tau_1 e_1^T) (I + \tau_1 e_1^T) =
  I - \tau_1 e_1^T + \tau_1 e_1^T - \tau_1 e_1^T \tau_1 e_1^T = I,
\]
since $e_1^T \tau_1$ (the first entry of $\tau_1$) is zero.  Therefore,
\[
  (I-\tau_1 e_1^T)^{-1} = (I+\tau_1 e_1^T)
\]
Similarly,
\[
  (I-\tau_2 e_2^T)^{-1} = (I+\tau_2 e_2^T)
\]
Thus,
\[
  L = (I+\tau_1 e_1^T)(I + \tau_2 e_2^T).
\]
Now, note that because $\tau_2$ is only nonzero in the third element,
$e_1^T \tau_2 = 0$; thus,
\begin{align*}
  L &= (I+\tau_1 e_1^T)(I + \tau_2 e_2^T) \\
    &= (I + \tau_1 e_1^T + \tau_2 e_2^T + \tau_1 (e_1^T \tau_2) e_2^T \\
    &= I + \tau_1 e_1^T + \tau_2 e_2^T \\
    &= \begin{bmatrix} 1 & 0 & 0 \\ 0 & 1 & 0 \\ 0 & 0 & 1 \end{bmatrix} +
       \begin{bmatrix} 0 & 0 & 0 \\ 2 & 0 & 0 \\ 3 & 0 & 0 \end{bmatrix} +
       \begin{bmatrix} 0 & 0 & 0 \\ 0 & 0 & 0 \\ 0 & 2 & 0 \end{bmatrix}
     = \begin{bmatrix} 1 & 0 & 0 \\ 2 & 1 & 0 \\ 3 & 2 & 1 \end{bmatrix}.
\end{align*}

The final factorization is
\[
A =
\begin{bmatrix} 1 & 4 & 7 \\ 2 & 5 & 8 \\ 3 & 6 & 10 \end{bmatrix} =
\begin{bmatrix} 1 & 0 & 0 \\ 2 & 1 & 0 \\ 3 & 2 & 1 \end{bmatrix}
\begin{bmatrix} 1 & 4 & 7 \\ 0 & -3 & -6 \\ 0 & 0 & 1 \end{bmatrix} = LU.
\]
Note that the subdiagonal elements of $L$ are easy to read off: for $j
> i$, $l_{ij}$ is the multiple of row $j$ that we subtract from row
$i$ during elimination.  This means that it is easy to read off the
subdiagonal entries of $L$ during the elimination process.
