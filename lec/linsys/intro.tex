\section{Introduction}

For the next few lectures, we will explore methods to solve linear
systems.  Our main tool will be the factorization $PA = LU$, where
$P$ is a permutation, $L$ is a unit lower triangular matrix, and $U$ is
an upper triangular matrix.  As we will see, the Gaussian elimination
algorithm learned in a first linear algebra class implicitly computes
this decomposition; but by thinking about the decomposition explicitly,
we can come up with other organizations for the computation.

We emphasize a few points up front:
\begin{itemize}
\item {\em Some matrices are singular.}  Errors in this part of the class
  often involve attempting to invert a matrix that has no inverse.
  A matrix does {\em not} have to be invertible to admit an LU factorization.
  We will also see more subtle problems from {\em almost} singular matrices.
\item {\em Some matrices are rectangular.}  In this part of the class,
  we will deal almost exclusively with square matrices; if a rectangular
  matrix shows up, we will try to be explicit about dimensions.  That
  said, LU factorization makes sense for rectangular matrices as well
  as for square matrices --- and it is sometimes useful.
\item {\em {\tt inv} is evil.}  The {\tt inv} command is one of the most
  abused commands in \matlab.  The \matlab\ backslash operator is the
  preferred way to solve a linear system absent other information:
\begin{lstlisting}
x = A \ b;  % Good
x = inv(A) * b;  % Evil
\end{lstlisting}
  Homework solutions that feature inappropriate explicit {\tt inv}
  commands {\em will} lose points.
\item {\em LU is not for linear solves alone.}  One can solve a variety
  of other interesting problems with an LU factorization.
\item {\em LU is not the only way to solve systems.}  Gaussian elimination
  and variants will be our default solver, but there are other solver
  methods that are appropriate for problems with more structure.
  We will touch on other methods throughout the class.
\end{itemize}

% - Never use inv!
% - Rectangular matrices
% - Singular matrices
% - Determinants and the like
% - Blocking and block solves
% - Graph theory and sparse solvers
% - Iterative refinement
