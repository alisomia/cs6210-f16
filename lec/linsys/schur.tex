\section{Schur complements}

The idea of expressing a step of Gaussian elimination as a low-rank
submatrix update turns out to be sufficiently useful that we give it
a name.  At any given step of Gaussian elimination, the trailing
submatrix is called a {\em Schur complement}.  We investigate the
structure of the Schur complements by looking at an $LU$
factorization in block 2-by-2 form:
\[
  \begin{bmatrix}
    A_{11} & A_{12} \\
    A_{21} & A_{22}
  \end{bmatrix} =
  \begin{bmatrix}
    L_{11} & 0 \\
    L_{21} & L_{22}
  \end{bmatrix}
  \begin{bmatrix}
    U_{11} & U_{12} \\
        0 & U_{22}
  \end{bmatrix} =
  \begin{bmatrix}
    L_{11} U_{11} & L_{11} U_{12} \\
    L_{21} U_{11} & L_{22} U_{22} + L_{21} U_{12}
  \end{bmatrix}.
\]
We can compute $L_{11}$ and $U_{11}$ as $LU$ factors of the leading
sub-block $A_{11}$, and
\begin{align*}
  U_{12} &= L_{11}^{-1} A_{12} \\
  L_{21} &= A_{21} U_{11}^{-1}.
\end{align*}
What about $L_{22}$ and $U_{22}$?  We have
\begin{align*}
  L_{22} U_{22}
  &= A_{22} - L_{21} U_{12} \\
  &= A_{22} - A_{21} U_{11}^{-1} L_{11}^{-1} A_{12} \\
  &= A_{22} - A_{21} A_{11}^{-1} A_{12}.
\end{align*}
This matrix $S = A_{22} - A_{21} A_{11}^{-1} A_{12}$ is the block analogue
of the rank-1 update computed in the first step of the standard
Gaussian elimination algorithm.

For our purposes, the idea of a Schur complement is important because
it will allow us to write blocked variants of Gaussian elimination ---
an idea we will take up in more detail shortly.
But the Schur complement actually has meaning beyond being a matrix that
mysteriously appears as a by-product of Gaussian elimination.
In particular, note that if $A$ and $A_{11}$ are both invertible,
then
\[
  \begin{bmatrix} A_{11} & A_{12} \\ A_{21} & A_{22} \end{bmatrix}
  \begin{bmatrix} X \\ S^{-1} \end{bmatrix} =
  \begin{bmatrix} 0 \\ I \end{bmatrix},
\]
i.e. $S^{-1}$ is the $(2,2)$ submatrix of $A^{-1}$.
