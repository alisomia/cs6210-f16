\section{Low-rank updates and bordered systems}

We have already discussed block elimination: given a system
\[
  \begin{bmatrix} A & B \\ C & D \end{bmatrix}
  \begin{bmatrix} x \\ y \end{bmatrix} =
  \begin{bmatrix} f \\ g \end{bmatrix}
\]
we can compute
\begin{align*}
  S &= D-C A^{-1} B \\
  S y &= g-C A^{-1} f \\
  A x &= f-By.
\end{align*}
If $A$ has some structure such that solves with $A$ are simple,
we may want to use this block solve structure rather than
forming and factoring the bordered system.

Let us now consider a closely-related problem: suppose we want
to solve
\[
  (A+UW^T) x = f.
\]
This looks like a problem, but what happens if we define an
intermediate variable $y = W^T x$?  If we put the original
equation in terms of $x$ and $y$, and we add the equation
relating $x$ and $y$, we get a bordered system
\[
  \begin{bmatrix} A & U \\ W^T & -I \end{bmatrix}
  \begin{bmatrix} x \\ y \end{bmatrix} =
  \begin{bmatrix} f \\ 0 \end{bmatrix}.
\]
Solving this bordered system by block elimination and
collapsing away all the intermediate variables gives us
\begin{align*}
  x &= A^{-1} \left( f - U y \right) \\
    &= A^{-1} \left( f - U S^{-1} (-W^T A^{-1} f) \right) \\
    &= \left( A^{-1} - A^{-1} U (I+W^T U V)^{-1} W^T A^{-1} \right) f.
\end{align*}
We can re-interpret this as the {\em Sherman-Morrison-Woodbury formula}:
\[
  (A+UW^T)^{-1} = A^{-1} - A^{-1} U (I+W^T A^{-1} U)^{-1} W^T A^{-1}.
\]
The bordered system formulation (together with iterative refinement
to clean up potential issues with instability) is quite useful as a
solver.  I prefer it to the presentation of the SMW formula as a
{\em fait accompli}.
