\section{Similarity transforms}

When we talked about least squares problems, we spent some time
discussing the transformations that preserve the Euclidean norm:
orthogonal transformations.  It is worth spending a moment now to give
a name to the transformations that preserve the eigenvalue structure
of a matrix.  These are {\em similarity} transformations.

Suppose $A \in \bbC^{n \times n}$ is a square matrix, and $X \in
\bbC^{n \times n}$ is invertible.  Then the matrix $X A X^{-1}$ is
said to be {\em similar} to $A$, and the mapping from $A$ to $X A
X^{-1}$ is a {\em similarity transformation}.  If $A$ is the matrix for
an operator from $\bbC^n$ onto itself, then $X A X^{-1}$ is the matrix
for the same operator in a different basis.  The eigenvalues and the
Jordan block structure of a matrix are preserved under similarity, and
the matrix $X$ gives a relationship between the eigenvectors of $A$
and those of $X A X^{-1}$.  Note that this goes both ways: two
matrices have the same eigenvalues and Jordan block structure iff they
are similar.
