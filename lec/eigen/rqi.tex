\section{Changing shifts}

If we know a shift $\sigma$ that is close to a desired eigenvalue,
the shift-invert power method may be a reasonable method.  But even
with a good choice of shift, this method converges at best linearly
(i.e. the error goes down by a constant factor at each step).
We can do better by choosing a shift {\em dynamically}, so that as
we improve the eigenvector, we also get a more accurate shift.

Suppose $\hat{v}$ is an approximate eigenvector for $A$, i.e.
we can find some $\hat{\lambda}$ so that
\begin{equation} \label{rq-deriv-approx}
  A \hat{v} - \hat{v} \hat{\lambda} \approx 0.
\end{equation}
The choice of corresponding approximate eigenvalues is not so clear,
but a reasonable choice (which is always well-defined when $\hat{v}$
is nonzero) comes from multiplying (\ref{rq-deriv-approx}) by $\hat{v}^*$
and changing the $\approx$ to an equal sign:
\[
  \hat{v}^* A \hat{v} - \hat{v}^* \hat{v} \hat{\lambda} = 0.
\]
The resulting eigenvalue approximation $\hat{\lambda}$ is
the {\em Rayleigh quotient}:
\[
  \hat{\lambda} = \frac{ \hat{v}^* A \hat{v} }{ \hat{v}^* \hat{v} }.
\]

If we dynamically choose shifts for shift-invert steps using
Rayleigh quotients, we get the {\em Rayleigh quotient iteration}:
\begin{align*}
  \lambda_{k+1} &= \frac{v^{(k)\,*} A v^{(k)}}{v^{(k)\,*} v^{(k)}} \\
  v^{(k+1)} &=
    \frac{ (A-\lambda_{k+1})^{-1} v^{(k)} }
         {\| (A-\lambda_{k+1})^{-1} v^{(k)} \|_2}
\end{align*}
Unlike the power method, the Rayleigh quotient iteration has locally
quadratic convergence --- so once convergence sets in, the number of
correct digits roughly doubles from step to step.  We will return to
this method later when we discuss symmetric matrices, for which
the Rayleigh quotient iteration has locally {\em cubic} convergence.
