\section{Inverse iteration and the QR method}

When we discussed the power method, we found that we could improve
convergence by a spectral transformation that mapped the eigenvalue we
wanted to something with large magnitude (preferably much larger than
the other eigenvalues).  This was the {\em shift-invert} strategy.
We already know there is a connection leading from the power method
to orthogonal iteration to the QR method, which we can summarize with
a small number of formulas.  Let us see if we can follow the same
path to uncover a connection from inverse iteration (the power method
with $A^{-1}$, a special case of shift-invert in which the shift is zero) to QR.
If we call the orthogonal factors
in orthogonal iteration  $\uQ^{(k)}$ ($\uQ^{(0)} = I$) and the iterates
in QR iteration $A^{(k)}$, we have
\begin{align}
  A^{k}   &= \uQ^{(k)} \uR^{(k)} \label{orth-iter-rel} \\
  A^{(k)} &= (\uQ^{(k)})^* A (\uQ^{(k)}).
\end{align}
In particular, note that because $R^{(k)}$ are upper triangular,
\[
  A^{k} e_1 = (\uQ^{(k)} e_1) r^{(k)}_{11};
\]
that is, the first column of $\uQ^{(k)}$ corresponds to the $k$th
step of power iteration starting at $e_1$.  What happens when we
consider negative powers of $A$?  Inverting (\ref{orth-iter-rel}),
we find
\[
  A^{-k} = (\uR^{(k)})^{-1} (\uQ^{(k)})^*
\]
The matrix $\tilde{R}^{(k)} = (\uR^{(k)})^{-1}$ is again upper triangular;
and if we look carefully, we can see in this fact another power iteration:
\[
  e_n^* A^{-k} = e_n^* \tilde{R}^{(k)} (\uQ^{(k)})^*
              = \tilde{r}^{(k)}_{nn} (\uQ^{(k)} e_n)^*.
\]
That is, the last column of $\uQ^{(k)}$ corresponds to a power iteration
converging to a {\em row} eigenvector of $A^{-1}$.
